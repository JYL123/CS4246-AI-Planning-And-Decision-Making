\documentclass[12pt]{article}
\usepackage{amsmath}
\usepackage[linesnumbered,ruled]{algorithm2e}
\usepackage[noend]{algpseudocode}
\usepackage{graphicx}
\usepackage{hyperref}
\usepackage{tikz,forest}
\usetikzlibrary{arrows.meta}
\usepackage{forest}
\usepackage{adjustbox}
\usepackage{skmath}
\usepackage{inputenc}
\usepackage{amsmath,amsthm,amssymb}
 \usepackage{tabu}
 \usepackage{graphicx}
\usepackage{wrapfig}
\graphicspath{ {images/} }
\usepackage{tcolorbox}
\newtheorem*{remark}{Remark}

\makeatletter
\def\BState{\State\hskip-\ALG@thistlm}
\makeatother

\forestset{
    .style={
        for tree={
            base=bottom,
            child anchor=north,
            align=center,
            s sep+=1cm,
    straight edge/.style={
        edge path={\noexpand\path[\forestoption{edge},thick,-{Latex}] 
        (!u.parent anchor) -- (.child anchor);}
    },
    if n children={0}
        {tier=word, draw, thick, rectangle}
        {draw, diamond, thick, aspect=2},
    if n=1{%
        edge path={\noexpand\path[\forestoption{edge},thick,-{Latex}] 
        (!u.parent anchor) -| (.child anchor) node[pos=.2, above] {Y};}
        }{
        edge path={\noexpand\path[\forestoption{edge},thick,-{Latex}] 
        (!u.parent anchor) -| (.child anchor) node[pos=.2, above] {N};}
        }
        }
    }
}


\begin{document}

\noindent
{\large{ \textbf{Deterministic Planning} }}\\

\noindent
\textsl{1. Classical planning}
\begin{itemize}
\item To scale up planning, a factored representation is usually used. 
\item Planning domain definition language (PDDL):
\begin{itemize}
\item \textbf{State}: a conjunction of fluents
\begin{itemize}
\item Fluents are state variables, representing variable that can change over time
\item Fluents are ground boolean variable
\item Example: At(Truck, Melbourne)
\item Data semantics are used:
\begin{itemize}
\item Closed world assumption: fluents not mentioned are false
\item Unique name assumption: $Truck_1$ and $Truck_2$ are distinct
\end{itemize}
\item The following are not allowed:
\begin{itemize}
\item $At(x, y) \gets$ because it is not grounded
\item $\neg Poor \gets$ because it is a negation
\item $At(Father(Fred), Sydney) \gets$ because PDDL does not allow functions   
\end{itemize}
\item Think of states as a set of fluents, manipulated using set operations
\end{itemize}

\item \textbf{Action}: PDDL specifies the result of an action in terms of what changes. What is left remains unchanged.
\begin{itemize}
\item A set of action is specified by an \textbf{action schema}. A schema consists of an \textbf{action name}, a lists of all \textbf{ variables} used, a \textbf{precondition} and an \textbf{effect} 
\item The precondition and effects are conjunctions of literals 
\item The result of executing action $a$ in state $s$ is a state $s'$ which is a set of fluents formed as follows:
\begin{itemize}
\item Start from $s$
\item Remove fluents that appear in the action's effect as negative literals: \textbf{delete list}
\item Add fluents that appear in the action's effect as positive literals: \textbf{add list} \\

$RESULT(s, a) = (s - DEL(a)) \cap ADD(a)$
\end{itemize}
\end{itemize}
\item A set of action schemas defines a planning \textbf{domain}
\item A specific problem is defined by adding an initial state and a goal
\item The \textbf{initial state} is a conjunction of ground atoms
\item The \textbf{goal state} is a conjunction of literals 
\item PDDL does not allow quantifiers
\end{itemize}

\item Planning as state-space search 
\begin{itemize}
\item \textbf{Forward search}:
\begin{itemize}
\item Difficulties for forward search includes: 
\begin{itemize}
\item State space can be very large - exponential with the number of state variables
\item Action space can be very large
\end{itemize}
\item Forward search is hopeless without good heuristics
\end{itemize}

\item \textbf{Backward relevant-state search}:
\begin{itemize}
\item Only consider actions that are relevant to the goal, or current state
\item There is a set of relevant states to consider at each step
\item In backward search, we regress from a state description to a predecessor state description 
\begin{itemize}
\item Distinguish between state and description: in a state, every variable is assigned a value(True/False). For a ground fluents, there are $2^n$ ground states. For n ground fluents, there are $3^n$ descriptions, each fluent can be positive, negative or not mentioned.
\item Example: $\neg Poor \land  Famous$ describes states where $Poor$ is false and $Famous$ is true, but other unmentioned fluents can have any values
\end{itemize}
\item Given a goal $g$ and action $a$, regression from $g$ over $a$ gives description $g'$:\\

$g' = (g - Add(a) \cap Precond(a))$
\end{itemize}
\end{itemize}
\item Heuristics for planning 
\begin{itemize}
\item \textbf{Ignore ore-conditions heuristic}: drops all pre-conditions
\begin{itemize}
\item Solves relaxed problem $\gets$ Admissible
\item Any single goal fluent achievable in one step
\item Number of steps is roughly number of unsatisfied  goals, except some actions can satisfy multiple goals, some may undo some goals.
\end{itemize}
\item \textbf{Ignore delete lists heuristic}
\begin{itemize}
\item Ignoring delete lists allows monotonic progress towards goal
\end{itemize}
\item \textbf{Problem decomposition}
\begin{itemize}
\item Divide goal into subgoals, solve subgoals, then combine them
\item If each subproblem uses an admissible heuristic, taking max is admissible
\item Assume subgoal independent: sum the cost of solving each subgoal 
\begin{itemize}
\item Solution optimistic when there is negative interaction: action in a one subplan deletes goal in another subplan (thus true cost $>$ sum of cost of subplans) $\gets$ admissible
\item Solution pessimistic when there is positive interaction: action in one subplan achieves goals in another subplan $\gets$ not admissible
\item If admissible, sum is better than max 
\item Admissible if an optimal solution is reuired
\end{itemize}

\end{itemize}
\end{itemize}
\end{itemize}

\noindent
\textsl{2. SATPLAN}
\begin{itemize}
\item One way to do planning is to transform the planning problem into a \textbf{Boolean satisfiability (SAT)} problem, and solve with a SAT solver
\item Solving SAT requires finding an assignment to variables that will make a Boolean formula true, or declare that no assignment exists
\item SAT solvers usually takes input in \textbf{conjunctive normal form (CNF)} formulas:
\begin{itemize}
\item A CNF formula is a conjunction of clauses
\item A clause is a disjunction of literals 
\item A literals is a variable or its negation
\item Any boolean formula can be converted to CNF
\end{itemize}
\item Translat PDDL into SAT
\begin{itemize}
\item Propositionalize the actions: replace each action schema with set of ground actions by substituting constraints for each variable
\item Define initial state: asset $F^0$ for every fluent $F$ in initial state and $\neg F^0$ for every fluent not in the initial state
\item Propositionalize the goal: each goal is a conjunction constructs a disjunction over all possible ground conjunctions obtained by replacing the variable with constraints
\item Add successor-state axioms: for each fluent F, add axiom of the form:\\

$F^{t+1} \Leftrightarrow ActionCauseF^t \lor (F^t \land \neg ActionCauseNotF^t)$

\item Add preconditoin axioms: for each ground action A, add axiom $A^t \Rightarrow PRE(A)^t$ (if a action is taken at time t, its preconds must have been true)
\item Add action exclusion axioms: say that every action is distinct from every other actions, i.e. only one action is allowed at each time step. 
\begin{itemize}
\item To do this, for every pair of actions $A_i^t$ and $A_j^t$: add mutual exclusion constraint $\neg A_i^t \lor A_j^t$
\item If we want to allow parallel actions, add mutual exclusion only if pair of action really interfere with each other
\end{itemize}
\end{itemize}
\end{itemize}
\pagebreak
\begin{itemize}
\begin{algorithm}
    \SetKwInOut{Input}{Input}
    \SetKwInOut{Output}{Output}

    \underline{function SATPLAN(init, transition, goal, $T_{max}$) returns action or failure} \\
    \Input{init, transition, goal consisitute a description of the problem;  $T_{max}$, an upper limit for plan length}
    \Output{$solution / failure$}\
    
      \For{$t = 0;\ i < T_{max};\ t = i + 1$}{
    cnf $\gets$ TRANSLATE-TO-SAT(init, transition, goal, t)\;
    model $\gets$ SAT-SOLVER(cnf)
    \If{$model$ is not null}
      {
        return EXTRACT-SOLUTION(model)
      }
  }
      return failure
      \caption{SAT plan}
\end{algorithm}

\item The number of steps required is not known in advance: try every value for $T$ up to $T_{max}$
\item Instead of Boolean SAT, can also encode problem as constraint satisfaction problem (CSP). Similar, but variable need not be binary
\item PlanSAT and Bounded PlanSAT asks whether there is a solution of length k or less. Both problems are decidable for classical planning as the number of states is finite (there is no function allowed)
\begin{itemize}
\item If function symbols are added, number of states becomes infinite and PlanSAT becomes semidecidable (if there exist a solution, return the solution but may not terminate when no solution exists)
\end{itemize}
\end{itemize}






\end{document}